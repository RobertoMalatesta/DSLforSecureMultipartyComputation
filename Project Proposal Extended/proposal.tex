%%%%%%%%%%%%%%%%%%%%%%%%%%%%%%%%%%%%%%%%%
% Short Sectioned Assignment
% LaTeX Template
% Version 1.0 (5/5/12)
%
% This template has been downloaded from:
% http://www.LaTeXTemplates.com
%
% Original author:
% Frits Wenneker (http://www.howtotex.com)
%
% License:
% CC BY-NC-SA 3.0 (http://creativecommons.org/licenses/by-nc-sa/3.0/)
%
%%%%%%%%%%%%%%%%%%%%%%%%%%%%%%%%%%%%%%%%%

%----------------------------------------------------------------------------------------
%	PACKAGES AND OTHER DOCUMENT CONFIGURATIONS
%----------------------------------------------------------------------------------------

\documentclass[paper=a4, fontsize=11pt]{scrartcl} % A4 paper and 11pt font size

\usepackage[T1]{fontenc} % Use 8-bit encoding that has 256 glyphs
\usepackage{fourier} % Use the Adobe Utopia font for the document - comment this line to return to the LaTeX default
\usepackage[english]{babel} % English language/hyphenation
\usepackage{amsmath,amsfonts,amsthm} % Math packages
\usepackage{ulem}
\usepackage{lipsum} % Used for inserting dummy 'Lorem ipsum' text into the template
\usepackage[ pdftex, plainpages = false, pdfpagelabels, 
                 pdfpagelayout = useoutlines,
                 bookmarks,
                 bookmarksopen = true,
                 bookmarksnumbered = true,
                 breaklinks = true,
                 linktocpage,
                 pagebackref = false,
                 colorlinks = true,
                 linkcolor = blue,
                 urlcolor  = blue,
                 citecolor = red,
                 anchorcolor = green,
                 hyperindex = true,
                 hyperfigures
                 ]{hyperref} 

\usepackage{sectsty} % Allows customizing section commands
\allsectionsfont{\centering \normalfont\scshape} % Make all sections centered, the default font and small caps

\usepackage{fancyhdr} % Custom headers and footers
\pagestyle{fancyplain} % Makes all pages in the document conform to the custom headers and footers
\fancyhead{} % No page header - if you want one, create it in the same way as the footers below
\fancyfoot[L]{} % Empty left footer
\fancyfoot[C]{} % Empty center footer
\fancyfoot[R]{\thepage} % Page numbering for right footer
\renewcommand{\headrulewidth}{0pt} % Remove header underlines
\renewcommand{\footrulewidth}{0pt} % Remove footer underlines
\setlength{\headheight}{1pt} % Customize the height of the header

\numberwithin{equation}{section} % Number equations within sections (i.e. 1.1, 1.2, 2.1, 2.2 instead of 1, 2, 3, 4)
\numberwithin{figure}{section} % Number figures within sections (i.e. 1.1, 1.2, 2.1, 2.2 instead of 1, 2, 3, 4)
\numberwithin{table}{section} % Number tables within sections (i.e. 1.1, 1.2, 2.1, 2.2 instead of 1, 2, 3, 4)

\setlength\parindent{0pt} % Removes all indentation from paragraphs - comment this line for an assignment with lots of text

%----------------------------------------------------------------------------------------
%	TITLE SECTION
%----------------------------------------------------------------------------------------

\newcommand{\horrule}[1]{\rule{\linewidth}{#1}} % Create horizontal rule command with 1 argument of height

\title{	
\vspace*{-2cm}
\normalfont \normalsize 
\textsc{Program Analysis and Understanding} \\ [10pt] % Your university, school and/or department name(s)
\horrule{0.2pt} \\[0.2cm] % Thin top horizontal rule
\huge Embedding MPC into {\tt C++} \\ % The assignment title
\horrule{2pt} \\[0.2cm] % Thick bottom horizontal rule
}
\author{Nikolaos Kofinas} % Your name
\date{}%\normalsize\today} % Today's date or a custom date
\begin{document}

\maketitle % Print the title

%----------------------------------------------------------------------------------------
%	PROBLEM 1
%----------------------------------------------------------------------------------------
\section{Motivation}

Various modern applications use Multiparty Computations to join the data from various parties/users in order to extract a result over the total data. These computation are trivial but now-days most of the users request a way to do them while they are keeping their data private without passing them to other parties or a trusted authority. The theory for Secure Multiparty Computations exists but there is no easy way for programmers to implement them in a popular language like {\tt C}, {\tt C++}, {\tt Java}, {\tt Python}, etc. Resent works tried to create new languages that let the programmer to write code which have the capability to handle both local and share computations. Unfortunately, these languages are independent and their is no easy way to embed them in a popular language. 

\section{Related Work}

\noindent
\textbf{Fairplay}: Fairplay~\cite{Fairplay} project created a Domain Specific Language which can be compiled directly into boolean circuits which are used in order to run this code in a Secret Multiparty Computations engine. The syntax of Fairplay is similar to the syntax of the Pascal programming language and thus it is not familiar to most of the programmers. Additionally, Fairplay allow the user to write only share code and thus the local part of the code must be written in a different language. 

\noindent
\textbf{PCF}: The language that is part of the Portable Circuit Format~\cite{Ansi-c} project is similar to ansi-C and thus most of the programmers can used it. While the idea of this project to create a DSL which syntax is very close to an actual language, is has two disadvantages. First of all, it is designed only to work with two parties and thus it cannot be scaled to n parties and secondly it has the same problem as Fairplay, the programmers needs a host language in order to write the local code.

\noindent
\textbf{Wysteria}: Wysteria~\cite{Wysteria} is a Domain Specific Language which can provide to the programmer the power to write local and shared computations inside the same code. This is a significant advantage compared to the previous two projects because the programmer does not need a host language. On the other hand there are two disadvantages with this dsl, firstly the syntax of this language is very similar to oCaml and thus it can be strange for some of the programmers and secondly, it provides to the  programmer only basic programming capabilities.
 
\section{Proposal}

In this project, we would like to embed Secure Multiparty Computations inside the {\tt C++} programming language. We chose {\tt C++} because it is a very popular languages and it used to write all sort of applications. Thus, if we embed MPC inside {\tt C++} will be possible to create any kind of applications which is impossible with the other, independent languages. Some examples of such applications are: a poker application with GUI which can be used from real users, an application that requires some dynamic input from the user, etc.

The first stage of this project is to formalize the problem and see what exactly we need to embed into {\tt C++}  During this step we need to find out what functionality do we need from our DSL. The current idea is that the DSL will be used only to write a function which will contain all the secret multiparty computations and it will be callable from anywhere inside the {\tt C++} code. Also, we need to find out what types, operators, and statements we need to include in our DSL. Finally, ew must specify the syntax of the DSL which we want to be as similar to the syntax of {\tt C++} as possible.

The second part will be to implement the interpreter for our language inside {\tt C++}. Currently, the idea is to use the {\tt C++} meta-programming language together with the boost mpl library. The implementation of our interpreter for is the most difficult part of this task and thus our first implementation will not contain the conversion to boolean circuits in order to make the problem simpler. 

The third part will be to add the translation to boolean circuits inside our interpreter. Along side with the code that will create the circuits we must add the appropriate code that will call the GMW~\cite{GMW} server who will handle all the communication between the parties.

\section{Language}
We started to writing down the types, statements and operations that our DSL will have:
\begin{itemize}
\item \textbf{Types}: bool, int (16 or 32), arrays of int or bool
\item \textbf{Operators}: <, >, ==, !=, !, +, -, =, \&, |, \&\&, ||, $\wedge$
\item \textbf{Statements}: if then else
\end{itemize}

\noindent
\textbf{Types}:
As we can see, our DSL has only two types bool and int. The reason behind this restriction is that the translation to boolean circuits is very difficult for float and double types. Also, if we allow strings to our language then their length must be the same for all parties and, at the time, it is simpler to let this type outside of our language.

\textbf{Operators}: Our language support almost all the basic arithmetic and 
logic operations with the exception of two arithmetic operators '*' and '/'. The reason that our language will not have these operators is because it is difficult to create the appropriate boolean circuit for them.

\textbf{Statements}: The DSL will have only one statement, the if statement. This statements is easy to implement into boolean circuits compared to for, while, switch, and do-while. We will also need a for statement which can be unfold during compile time. Thus the initialization, condition, and increase part of the loop must be known during compile time in all parties (e.g. they would not contain any secret variable).

\section{Project Milestones}
In order to complete this project in the two month time window we decided to split it in small pieces and complete one at the time:
\begin{enumerate}
\item Milestone 1 (Deadline 30 Oct):
\begin{itemize}
\item Finalize the types, operations and statements of our language
\item Create the appropriate syntax for the language
\end{itemize}
\item Milestone 2 (Deadline 30 Nov):
\begin{itemize}
\item Design the interpreter of our DSL without the boolean conversion
\end{itemize}
\item Milestone 3 (Deadline 10 Dec):
\begin{itemize}
\item Finalize interpreter by creating the boolean conversion part
\end{itemize}
\end{enumerate}
As we can see we allocated one month for the second milestone because it is the most difficult. During the second milestone, we must found how to implement the DSL inside the C++ and also how to evaluate the DSL during compile time. The milestone 3 can be done in a sorter amount of time if the interpreter is functional. The hard deadline for the project is 15th of December and thus we have 5 additional days to allocate if a milestone stalls.
\begin{thebibliography}{9}
\bibitem{Fairplay}Ben-David, Assaf, Noam Nisan, and Benny Pinkas. ``FairplayMP: a system for secure multi-party computation.'' Proceedings of the 15th ACM conference on Computer and communications security. ACM, 2008.
\bibitem{Ansi-c} B. Kreuter, ahbi shelat, B. Mood, and K. Butler, ``PCF: A portable
circuit format for scalable two-party secure computation.'' in USENIX,
2013
\bibitem{Wysteria}
Rastogi, Aseem, Matthew A. Hammer, and Michael Hicks. ``Wysteria: A programming language for generic, mixed-mode multiparty computations.'' (2014).
\bibitem{GMW}S. G. Choi, K.-W. Hwang, J. Katz, T. Malkin, and D. Rubenstein,
“Secure multi-party computation of boolean circuits with applications to
privacy in on-line marketplaces,” 2011, \url{http://eprint.iacr.org/}
\end{thebibliography}
\end{document}
